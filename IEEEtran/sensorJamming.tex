\documentclass[conference]{IEEEtran}
% Some Computer Society conferences also require the compsoc mode option,
% but others use the standard conference format.
%
% Some very useful LaTeX packages include:
% (uncomment the ones you want to load)


% *** MISC UTILITY PACKAGES ***
%
%\usepackage{ifpdf}
% Heiko Oberdiek's ifpdf.sty is very useful if you need conditional
% compilation based on whether the output is pdf or dvi.
% usage:
% \ifpdf
%   % pdf code
% \else
%   % dvi code
% \fi
% The latest version of ifpdf.sty can be obtained from:
% http://www.ctan.org/pkg/ifpdf
% Also, note that IEEEtran.cls V1.7 and later provides a builtin
% \ifCLASSINFOpdf conditional that works the same way.
% When switching from latex to pdflatex and vice-versa, the compiler may
% have to be run twice to clear warning/error messages.






% *** CITATION PACKAGES ***
%
%\usepackage{cite}
% cite.sty was written by Donald Arseneau
% V1.6 and later of IEEEtran pre-defines the format of the cite.sty package
% \cite{} output to follow that of the IEEE. Loading the cite package will
% result in citation numbers being automatically sorted and properly
% "compressed/ranged". e.g., [1], [9], [2], [7], [5], [6] without using
% cite.sty will become [1], [2], [5]--[7], [9] using cite.sty. cite.sty's
% \cite will automatically add leading space, if needed. Use cite.sty's
% noadjust option (cite.sty V3.8 and later) if you want to turn this off
% such as if a citation ever needs to be enclosed in parenthesis.
% cite.sty is already installed on most LaTeX systems. Be sure and use
% version 5.0 (2009-03-20) and later if using hyperref.sty.
% The latest version can be obtained at:
% http://www.ctan.org/pkg/cite
% The documentation is contained in the cite.sty file itself.






% *** GRAPHICS RELATED PACKAGES ***
%
\ifCLASSINFOpdf
  % \usepackage[pdftex]{graphicx}
  % declare the path(s) where your graphic files are
  % \graphicspath{{../pdf/}{../jpeg/}}
  % and their extensions so you won't have to specify these with
  % every instance of \includegraphics
  % \DeclareGraphicsExtensions{.pdf,.jpeg,.png}
\else
  % or other class option (dvipsone, dvipdf, if not using dvips). graphicx
  % will default to the driver specified in the system graphics.cfg if no
  % driver is specified.
  % \usepackage[dvips]{graphicx}
  % declare the path(s) where your graphic files are
  % \graphicspath{{../eps/}}
  % and their extensions so you won't have to specify these with
  % every instance of \includegraphics
  % \DeclareGraphicsExtensions{.eps}
\fi
% graphicx was written by David Carlisle and Sebastian Rahtz. It is
% required if you want graphics, photos, etc. graphicx.sty is already
% installed on most LaTeX systems. The latest version and documentation
% can be obtained at: 
% http://www.ctan.org/pkg/graphicx
% Another good source of documentation is "Using Imported Graphics in
% LaTeX2e" by Keith Reckdahl which can be found at:
% http://www.ctan.org/pkg/epslatex
%
% latex, and pdflatex in dvi mode, support graphics in encapsulated
% postscript (.eps) format. pdflatex in pdf mode supports graphics
% in .pdf, .jpeg, .png and .mps (metapost) formats. Users should ensure
% that all non-photo figures use a vector format (.eps, .pdf, .mps) and
% not a bitmapped formats (.jpeg, .png). The IEEE frowns on bitmapped formats
% which can result in "jaggedy"/blurry rendering of lines and letters as
% well as large increases in file sizes.
%
% You can find documentation about the pdfTeX application at:
% http://www.tug.org/applications/pdftex





% *** MATH PACKAGES ***
%
%\usepackage{amsmath}
% A popular package from the American Mathematical Society that provides
% many useful and powerful commands for dealing with mathematics.
%
% Note that the amsmath package sets \interdisplaylinepenalty to 10000
% thus preventing page breaks from occurring within multiline equations. Use:
%\interdisplaylinepenalty=2500
% after loading amsmath to restore such page breaks as IEEEtran.cls normally
% does. amsmath.sty is already installed on most LaTeX systems. The latest
% version and documentation can be obtained at:
% http://www.ctan.org/pkg/amsmath





% *** SPECIALIZED LIST PACKAGES ***
%
%\usepackage{algorithmic}
% algorithmic.sty was written by Peter Williams and Rogerio Brito.
% This package provides an algorithmic environment fo describing algorithms.
% You can use the algorithmic environment in-text or within a figure
% environment to provide for a floating algorithm. Do NOT use the algorithm
% floating environment provided by algorithm.sty (by the same authors) or
% algorithm2e.sty (by Christophe Fiorio) as the IEEE does not use dedicated
% algorithm float types and packages that provide these will not provide
% correct IEEE style captions. The latest version and documentation of
% algorithmic.sty can be obtained at:
% http://www.ctan.org/pkg/algorithms
% Also of interest may be the (relatively newer and more customizable)
% algorithmicx.sty package by Szasz Janos:
% http://www.ctan.org/pkg/algorithmicx




% *** ALIGNMENT PACKAGES ***
%
%\usepackage{array}
% Frank Mittelbach's and David Carlisle's array.sty patches and improves
% the standard LaTeX2e array and tabular environments to provide better
% appearance and additional user controls. As the default LaTeX2e table
% generation code is lacking to the point of almost being broken with
% respect to the quality of the end results, all users are strongly
% advised to use an enhanced (at the very least that provided by array.sty)
% set of table tools. array.sty is already installed on most systems. The
% latest version and documentation can be obtained at:
% http://www.ctan.org/pkg/array


% IEEEtran contains the IEEEeqnarray family of commands that can be used to
% generate multiline equations as well as matrices, tables, etc., of high
% quality.




% *** SUBFIGURE PACKAGES ***
%\ifCLASSOPTIONcompsoc
%  \usepackage[caption=false,font=normalsize,labelfont=sf,textfont=sf]{subfig}
%\else
%  \usepackage[caption=false,font=footnotesize]{subfig}
%\fi
% subfig.sty, written by Steven Douglas Cochran, is the modern replacement
% for subfigure.sty, the latter of which is no longer maintained and is
% incompatible with some LaTeX packages including fixltx2e. However,
% subfig.sty requires and automatically loads Axel Sommerfeldt's caption.sty
% which will override IEEEtran.cls' handling of captions and this will result
% in non-IEEE style figure/table captions. To prevent this problem, be sure
% and invoke subfig.sty's "caption=false" package option (available since
% subfig.sty version 1.3, 2005/06/28) as this is will preserve IEEEtran.cls
% handling of captions.
% Note that the Computer Society format requires a larger sans serif font
% than the serif footnote size font used in traditional IEEE formatting
% and thus the need to invoke different subfig.sty package options depending
% on whether compsoc mode has been enabled.
%
% The latest version and documentation of subfig.sty can be obtained at:
% http://www.ctan.org/pkg/subfig




% *** FLOAT PACKAGES ***
%
%\usepackage{fixltx2e}
% fixltx2e, the successor to the earlier fix2col.sty, was written by
% Frank Mittelbach and David Carlisle. This package corrects a few problems
% in the LaTeX2e kernel, the most notable of which is that in current
% LaTeX2e releases, the ordering of single and double column floats is not
% guaranteed to be preserved. Thus, an unpatched LaTeX2e can allow a
% single column figure to be placed prior to an earlier double column
% figure.
% Be aware that LaTeX2e kernels dated 2015 and later have fixltx2e.sty's
% corrections already built into the system in which case a warning will
% be issued if an attempt is made to load fixltx2e.sty as it is no longer
% needed.
% The latest version and documentation can be found at:
% http://www.ctan.org/pkg/fixltx2e


%\usepackage{stfloats}
% stfloats.sty was written by Sigitas Tolusis. This package gives LaTeX2e
% the ability to do double column floats at the bottom of the page as well
% as the top. (e.g., "\begin{figure*}[!b]" is not normally possible in
% LaTeX2e). It also provides a command:
%\fnbelowfloat
% to enable the placement of footnotes below bottom floats (the standard
% LaTeX2e kernel puts them above bottom floats). This is an invasive package
% which rewrites many portions of the LaTeX2e float routines. It may not work
% with other packages that modify the LaTeX2e float routines. The latest
% version and documentation can be obtained at:
% http://www.ctan.org/pkg/stfloats
% Do not use the stfloats baselinefloat ability as the IEEE does not allow
% \baselineskip to stretch. Authors submitting work to the IEEE should note
% that the IEEE rarely uses double column equations and that authors should try
% to avoid such use. Do not be tempted to use the cuted.sty or midfloat.sty
% packages (also by Sigitas Tolusis) as the IEEE does not format its papers in
% such ways.
% Do not attempt to use stfloats with fixltx2e as they are incompatible.
% Instead, use Morten Hogholm'a dblfloatfix which combines the features
% of both fixltx2e and stfloats:
%
% \usepackage{dblfloatfix}
% The latest version can be found at:
% http://www.ctan.org/pkg/dblfloatfix




% *** PDF, URL AND HYPERLINK PACKAGES ***
%
%\usepackage{url}
% url.sty was written by Donald Arseneau. It provides better support for
% handling and breaking URLs. url.sty is already installed on most LaTeX
% systems. The latest version and documentation can be obtained at:
% http://www.ctan.org/pkg/url
% Basically, \url{my_url_here}.




% *** Do not adjust lengths that control margins, column widths, etc. ***
% *** Do not use packages that alter fonts (such as pslatex).         ***
% There should be no need to do such things with IEEEtran.cls V1.6 and later.
% (Unless specifically asked to do so by the journal or conference you plan
% to submit to, of course. )


% correct bad hyphenation here
\hyphenation{op-tical net-works semi-conduc-tor}
\usepackage[strings]{underscore}
\usepackage{enumitem}

\begin{document}
%
% paper title
% Titles are generally capitalized except for words such as a, an, and, as,
% at, but, by, for, in, nor, of, on, or, the, to and up, which are usually
% not capitalized unless they are the first or last word of the title.
% Linebreaks \\ can be used within to get better formatting as desired.
% Do not put math or special symbols in the title.
\title{Sensor Jamming Detection \\ and Mitigation Techniques}


% author names and affiliations
% use a multiple column layout for up to three different
% affiliations
\author{\IEEEauthorblockN{Dana Lombardi}
\IEEEauthorblockA{School of Engineering and\\Computer Science\\
Oakland University\\
Rochester, Michigan\\
Email: lombardd@mail.gvsu.edu}
\and
\IEEEauthorblockN{Justin Gluck}
\IEEEauthorblockA{School of Engineering and\\Computer Science\\
Oakland University\\
Rochester, Michigan\\
Email: justin@awesome.com}
\and
\IEEEauthorblockN{Brett McIsaac}
\IEEEauthorblockA{School of Engineering and\\Computer Science\\
Oakland University\\
Rochester, Michigan\\
Email: geekman3454@gmail.com}}

% conference papers do not typically use \thanks and this command
% is locked out in conference mode. If really needed, such as for
% the acknowledgment of grants, issue a \IEEEoverridecommandlockouts
% after \documentclass

% for over three affiliations, or if they all won't fit within the width
% of the page, use this alternative format:
% 
%\author{\IEEEauthorblockN{Michael Shell\IEEEauthorrefmark{1},
%Homer Simpson\IEEEauthorrefmark{2},
%James Kirk\IEEEauthorrefmark{3}, 
%Montgomery Scott\IEEEauthorrefmark{3} and
%Eldon Tyrell\IEEEauthorrefmark{4}}
%\IEEEauthorblockA{\IEEEauthorrefmark{1}School of Electrical and Computer Engineering\\
%Georgia Institute of Technology,
%Atlanta, Georgia 30332--0250\\ Email: see http://www.michaelshell.org/contact.html}
%\IEEEauthorblockA{\IEEEauthorrefmark{2}Twentieth Century Fox, Springfield, USA\\
%Email: homer@thesimpsons.com}
%\IEEEauthorblockA{\IEEEauthorrefmark{3}Starfleet Academy, San Francisco, California 96678-2391\\
%Telephone: (800) 555--1212, Fax: (888) 555--1212}
%\IEEEauthorblockA{\IEEEauthorrefmark{4}Tyrell Inc., 123 Replicant Street, Los Angeles, California 90210--4321}}




% use for special paper notices
%\IEEEspecialpapernotice{(Invited Paper)}




% make the title area
\maketitle

% As a general rule, do not put math, special symbols or citations
% in the abstract
\begin{abstract}
The abstract goes here.
\end{abstract}

% no keywords




% For peer review papers, you can put extra information on the cover
% page as needed:
% \ifCLASSOPTIONpeerreview
% \begin{center} \bfseries EDICS Category: 3-BBND \end{center}
% \fi
%
% For peerreview papers, this IEEEtran command inserts a page break and
% creates the second title. It will be ignored for other modes.
\IEEEpeerreviewmaketitle



\section{Introduction}
% no \IEEEPARstart
The autonomous vehicle or the 'self-driving car', once thought to be a distant technology of the future, is fast becoming a soon to be reality as a form of transportation. The technology promises to make driving safer, lessen traffic in busy cities and to improve upon the general driving experience, to name just a few of the benefits. \\
\indent One of the crucial aspects that make autonomous vehicles a possibility is sensor technology. Sensors allow a vehicle to become aware or "see" their surroundings. Sensor technology necessary for autonomous driving will likely vary from manufacturer to manufacturer but would probably include ultrasonic, radar, LIDAR, GPS, camera sensors and light sensors at the very least. These sensor types are not only limited to self-driving cars, this technology is used in many other areas such as unmanned aerial vehicles(UAVs) or drones, use a lot of similar technology in order to sense its surroundings and location. \\
\indent Autonomous vehicles must not only keep the passengers of the car safe but also not cause harm to any other cars, pedestrians or anything a vehicle could come into contact with. Due to this fact, there is a heavy burden placed upon this technology to function without fail. This means that sensor technology within the vehicle cannot be defective and this includes not being susceptible to malicious attacks or accidental attacks. One of the biggest threats sensors face in autonomous vehicles is \textit{sensor jamming}. \\
\indent It is necessary to first have a clear idea of what sensor jamming is. Sensor jamming is typically defined as any outside party or entity that is intentionally sending or causing interference to signals being sent or received to sensors during autonomous sensor communication\cite{2}\cite{7}. Although there are different types of jamming attacks that are possible, they will likely fall under this broad definition \cite{7}. Jamming is typically done with a device that will send signals flood the sensor with false data that misrepresents the surroundings or overwhelms the sensor causing it to fail. Additionally, jamming can be done through blocking the signals to and from the sensor. Various types of sensor jamming will be discussed in a later section.\\

\subsection{Mission Statement}
For this paper, our initial research goals were to determine possible sensor jamming \textit{detection} and \textit{mitigation} techniques for sensors within autonomous vehicles. Stopping or mitigating the effects of sensor jamming is crucial for functionality, safety, and widespread adoption of autonomous technology. Throughout our research, we wanted to also put into practice and perform testing and analysis of techniques ourselves in addition to analyzing other researched sensor jamming detection and mitigation techniques. \\

\subsection{Context of Research Area}
The focus and context in which we performed our research was on sensors and their usage within autonomous vehicles. Although many of the jamming detection and mitigation techniques are applicable to many types of sensors, we focused upon ultrasonic sensors, especially in our own testing and analysis.\\ \indent There are a few reasons for this, the first being that autonomous vehicles use ultrasonic sensors for close range detection of objects to assist with parking, adaptive cruise control and traffic jam assist\cite{10}. These are very important features to for safety in autonomous vehicles, as well newer non-autonomous vehicles that include some variation of these features, and if these are sensors are jammed, functionality is severely impacted. Failure can cause danger to passengers and vehicle surroundings. A second reason ultrasonic sensors had more focus was due to the relative low price and availability of these sensors to use for testing. In this paper, we will discuss further how an ultrasonic transducer(HC-SR04) used with an Arduino Uno was used during testing to gather data and provide a proof of concept to our hypothesis that jamming attacks that cause objects around a vehicle to be hidden from ultrasonic sensors can be done with cheap and readily available equipment.\\
\indent In addition to the testing that was done, several types of jamming detection and mitigation techniques were researched and analyzed. Detection and mitigation techniques studied were \textit{frequency hopping}\cite{2}\cite{4}, \textit{signal verification protocols}\cite{6}\cite{7}, and the \textit{filtering/extraction of jamming signals}\cite{4}\cite{5}.\\
\subsection{Terminology}
This section defines commonly used terminology throughout the paper.\\

\section{Sensor Jamming Attack Models}
There are various ways in which a sensor can be jammed. Because of this jamming attacks can be split up into subcategories. This section includes some of the most common types jamming came across throughout the research process.\\
\begin{description}

  \item[$\bullet$ Constant Jamming:] A constant jamming attack is one that operates by emitting a constant stream of random interfering signals the target sensor. This type of attack typically doesn't wait for any communication from the sensor, the attack is consistent and constant while it is happening. This type of attack can block any legitimate signal traffic from communicating with the sensor\cite{5}.\\
   \item[$\bullet$ Deceptive Jamming:] A jamming device tricks the sensor by sending packets or signals to the sensor. The sensor believes this to be a real packet and will switch into a receiving state. By continuously sending these normal signals to the sensor, the sensor is unable to send signals because it is falsely stuck in a receive state\cite{5}.\\
    \item[$\bullet$ Random Jamming:]  A random jamming attack will oscillate between states of sleeping and awake. When a random jamming device is awake, it can operate as a constant jammer or a deceptive jammer. This is done to conserve energy and not have the jamming device powered on at all times\cite{5}. \\
     \item[$\bullet$ Reactive Jamming:] The three previous jamming attacks typically target the entire frequency or channel the sensor is operating on. A reactive jamming attack is different in that it will remain idle while listening and scanning traffic until there is activity on the operating frequency band and the jammer will then be able to match and attack that sensor's signals\cite{2}\cite{5}.\\

\end{description}



\section{Related/Previous Works}
Orient readers with most relevant studies. \\
Explain how it's related to our approach\\
How our Study builds upon previous works.\\
There have been numerous studies on sensory jamming within autonomous technology. 
Three of the top studies include, frequency/channel hopping, signal verification algorithms, 
and filtering of the jamming signals. Each is discussed further below.\\

\section{Problem Overview}
\subsection{Problem Statement}
\subsection{Challenges of the Problem}
\subsubsection{Adversary Goals and Capabilities}
**Unsure where to fit this in as of now, maybe won't need it's own section**
\section{Our Contribution}
\subsection{Research Goals for Solutions}
1. Detect when sensor is being jammed
2. Immediately work to mitigate any jamming
3. Allow sensor to continue to function despite attacks
\subsection{Running our own Simulation/Testing}
1. Proof of Concept
2. Proof of Interference of sensors(does the way this works affect its reliability?)
\subsection{Analysis and Evaluation}
1.How do we sense an attack or interference is happening?
2. What is the best way to alert drivers?
3. What is a consequence of such an attack?

\section{Jamming Detection and Mitigation Techniques}
Description of different types of jamming detection and prevention from research. How we were able to simulate or evaluate it's effectiveness.
\subsection{Group Experimentation and Testing}
\subsubsection{Proof of Concept}
\subsubsection{Computer Experiment}
\subsubsection{Setup}
\subsubsection{Results}
\subsection{Frequency Hopping}
\textit{Frequency Hopping}(FH) is the process by which the sensor continuously changes or 'hops' the frequency channel signals are being sent or received on. Hopping patterns will be unknown to attackers and likely unpredictable as well\cite{5}. If FH is happening quickly, jamming devices will be unable to react quickly enough to match to the current operating frequency, thus evading any jammers that were scanning the sensor such as reactive jammers\cite{5}.\\
\indent FH can also help prevent jamming that happens when there is 'cross-talk' between sensors. This occurs when sensors of the same type, operating on the same frequency are close enough that they are sending signals to or receiving signals from the nearby sensors accidentally\cite{3}. This causes sensors to receive false data or accidentally jam other sensors. If these near-by sensors have signals FH enough it is unlikely they would interfere with each other. Meng. et. al. proposes a strategy to mitigate this type of jamming by transmitting ultrasonic signals in pseudo-random pulses within varying frequencies. The sensor then waits on an expected, returned signal corresponding to the pseudo-random pulse and frequency that was sent out\cite{3}.
\subsection{Signal Verification Algorithms}
-How is it detected
-How do we mitigate
-does sensor continue to function?
\subsubsection{Signal to Noise Ratio}
\subsubsection{Send to Delivery Ratio}
\subsubsection{Measuring Signal Strength}
\subsection{Filtering Techniques}
-How is it detected
-How do we mitigate
-does sensor continue to function?
\section{Security Analysis}
\subsection{Feasibility}
\subsection{Scalability}
\subsection{Effectiveness in Achieving Goals}
\section{Conclusion}
Best way to alert driver?

\section{Recommendation for Future Research Focus}

\section{Individual Contributions}




% conference papers do not normally have an appendix


% use section* for acknowledgment
\section*{Acknowledgment}


The authors would like to thank...

% An example of a floating figure using the graphicx package.
% Note that \label must occur AFTER (or within) \caption.
% For figures, \caption should occur after the \includegraphics.
% Note that IEEEtran v1.7 and later has special internal code that
% is designed to preserve the operation of \label within \caption
% even when the captionsoff option is in effect. However, because
% of issues like this, it may be the safest practice to put all your
% \label just after \caption rather than within \caption{}.
%
% Reminder: the "draftcls" or "draftclsnofoot", not "draft", class
% option should be used if it is desired that the figures are to be
% displayed while in draft mode.
%
%\begin{figure}[!t]
%\centering
%\includegraphics[width=2.5in]{myfigure}
% where an .eps filename suffix will be assumed under latex, 
% and a .pdf suffix will be assumed for pdflatex; or what has been declared
% via \DeclareGraphicsExtensions.
%\caption{Simulation results for the network.}
%\label{fig_sim}
%\end{figure}

% Note that the IEEE typically puts floats only at the top, even when this
% results in a large percentage of a column being occupied by floats.


% An example of a double column floating figure using two subfigures.
% (The subfig.sty package must be loaded for this to work.)
% The subfigure \label commands are set within each subfloat command,
% and the \label for the overall figure must come after \caption.
% \hfil is used as a separator to get equal spacing.
% Watch out that the combined width of all the subfigures on a 
% line do not exceed the text width or a line break will occur.
%
%\begin{figure*}[!t]
%\centering
%\subfloat[Case I]{\includegraphics[width=2.5in]{box}%
%\label{fig_first_case}}
%\hfil
%\subfloat[Case II]{\includegraphics[width=2.5in]{box}%
%\label{fig_second_case}}
%\caption{Simulation results for the network.}
%\label{fig_sim}
%\end{figure*}
%
% Note that often IEEE papers with subfigures do not employ subfigure
% captions (using the optional argument to \subfloat[]), but instead will
% reference/describe all of them (a), (b), etc., within the main caption.
% Be aware that for subfig.sty to generate the (a), (b), etc., subfigure
% labels, the optional argument to \subfloat must be present. If a
% subcaption is not desired, just leave its contents blank,
% e.g., \subfloat[].


% An example of a floating table. Note that, for IEEE style tables, the
% \caption command should come BEFORE the table and, given that table
% captions serve much like titles, are usually capitalized except for words
% such as a, an, and, as, at, but, by, for, in, nor, of, on, or, the, to
% and up, which are usually not capitalized unless they are the first or
% last word of the caption. Table text will default to \footnotesize as
% the IEEE normally uses this smaller font for tables.
% The \label must come after \caption as always.
%
%\begin{table}[!t]
%% increase table row spacing, adjust to taste
%\renewcommand{\arraystretch}{1.3}
% if using array.sty, it might be a good idea to tweak the value of
% \extrarowheight as needed to properly center the text within the cells
%\caption{An Example of a Table}
%\label{table_example}
%\centering
%% Some packages, such as MDW tools, offer better commands for making tables
%% than the plain LaTeX2e tabular which is used here.
%\begin{tabular}{|c||c|}
%\hline
%One & Two\\
%\hline
%Three & Four\\
%\hline
%\end{tabular}
%\end{table}








% trigger a \newpage just before the given reference
% number - used to balance the columns on the last page
% adjust value as needed - may need to be readjusted if
% the document is modified later
%\IEEEtriggeratref{8}
% The "triggered" command can be changed if desired:
%\IEEEtriggercmd{\enlargethispage{-5in}}

% references section

% can use a bibliography generated by BibTeX as a .bbl file
% BibTeX documentation can be easily obtained at:
% http://mirror.ctan.org/biblio/bibtex/contrib/doc/
% The IEEEtran BibTeX style support page is at:
% http://www.michaelshell.org/tex/ieeetran/bibtex/
%\bibliographystyle{IEEEtran}
% argument is your BibTeX string definitions and bibliography database(s)
%\bibliography{IEEEabrv,../bib/paper}
%
% <OR> manually copy in the resultant .bbl file
% set second argument of \begin to the number of references
% (used to reserve space for the reference number labels box)

\cite{1}
\cite{2}
\cite{3}
\cite{4}
\cite{5}
\cite{6}
\cite{7}
\cite{8}
\cite{9}
\cite{10}
\clearpage 

\bibliographystyle{IEEEtran}
\bibliography{References}





% that's all folks
\end{document}


